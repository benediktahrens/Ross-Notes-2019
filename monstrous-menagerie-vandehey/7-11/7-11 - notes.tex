
\documentclass{article}

\title{Monstrous Menagerie with Vandehey 7/11}
\author{Jason Schuchardt}
\date{\today}

\usepackage{jragonfyre}
\newcommand\cft[1]{\frac{1}{#1+}}
\newcommand\hoint{[0,1)}
\newcommand\recip[1]{\frac{1}{#1}}
\newcommand\cylset[1]{C_{[{#1}]}}


\begin{document}
\maketitle

We're proving the lemma from the end of last time.
\begin{proof}
    By Stirling's formula we have:
    \begin{align*}
    \binom{n}{cn} 
    &= \frac{n!}{(cn)!((1-c)n)!}
    \\
    &= \frac{\sqrt{2\pi n} \of*{\frac{n}{e}}^n \of*{1+O\of*{\frac{1}{n}}}}{
        \sqrt{2\pi cn} \of*{\frac{cn}{e}}^{cn} \of*{1+O\of*{\frac{1}{cn}}}
        \sqrt{2\pi (1-c)n} \of*{\frac{(1-c)n}{e}}^{(1-c)n} \of*{1+O\of*{\frac{1}{(1-c)n}}}
    }
    \\
    &= 
    \frac{1}{\sqrt{2\pi c(1-c)n}} 
    \braks*{c^{-c} (1-c)^{-(1-c)}}^n
    \frac{1+O\of*{\frac{1}{n}}}{
        \of*{1+O\of*{\frac{1}{cn}}}
        \of*{1+O\of*{\frac{1}{(1-c)n}}}
        }
    \end{align*}

    Then $O\of*{c\frac{1}{n}} = O\of*{\frac{1}{n}}$
    (note that equals signs don't represent equality
    for asymptotic notation, so this is an asymmetric
    statement).
    Also 
    \[\frac{1}{1+O\of*{\frac{1}{n}}} 
    = 1+O\of*{\frac{1}{n}},
    \]
    and 
    \[
    \of*{1+O\of*{\frac{1}{n}}}
    \of*{1+O\of*{\frac{1}{n}}}
    =1+O\of*{\frac{1}{n}},
    \]
    so we can reduce the last term in the product 
    to just $1+O(1/n)$, as desired.
\end{proof}

\begin{proposition}
    Let $k\in \NN$ be fixed, and let $\epsilon > 0$
    be sufficiently small in terms of $k$ and $b$.
    Then there exists $\delta > 0$, 
    $\delta=\delta(\epsilon,k)$ such that the number 
    of integers in $[1,m]$ that are not 
    $(\epsilon,k)$-normal is $O(m^{1-\delta)})$.
    Also there eixsts $\delta'=\delta'(\epsilon,k)>0$
    such that the number of base-$b$ strings of 
    length $\ell$ (including those starting with 
    $0$) that are not not $(\epsilon,k)$-normal is 
    $O\of*{b^{\ell(1-\delta')}}$.
\end{proposition}

\begin{proof}
    The first statement follows from the second 
    (exercise). For the second half, assume 
    $k=1$ and $\epsilon > 0$ is 
    sufficiently small. Let $d\in \set{0,1,\ldots, b-1}$
    be a digit, and consider how many strings of length
    $\ell$ have \emph{exactly} $m$ appearances of $d$.

    There are $\binom{\ell}{m}$ ways to pick the 
    positions for the $d$.
    Then there are $(b-1)^{\ell-m}$ possibilities
    for the remaining digits.

    So there are $\binom{\ell}{m} (b-1)^{\ell -m }$
    possible strings containing exactly $m$ appearances 
    of $d$.

    So the number of $(\epsilon,1)$-nonnormal strings of 
    length $\ell$ must be bounded by 
    \[
        b
        \of*{
            \sum_{m< \ell\of*{\frac{1}{b}-\epsilon}}
            \binom{\ell}{m}(b-1)^{\ell-m}
            +
            \sum_{m > \ell\of*{\frac{1}{b} + \epsilon}}
            \binom{\ell}{m}(b-1)^{\ell-m}
        }
    \]

    We will show how to get the desired bound on
    the first sum, and the second sum will be 
    similarly bounded.

    Consider 
    \[
        \binom{\ell}{m} (b-1)^{\ell-m}
        = \frac{\ell!}{m!(\ell-m)!}(b-1)^{\ell-m}
    \]
    In going from the $m$th term to the $(m+1)$th 
    term, we multiply by 
    \[\frac{\ell-m}{(m+1)(b-1)}.\]

    Thus the terms are increasing provided 
    \[\frac{\ell-m}{(m+1)(b-1)} > 1,\]
    or
    \begin{align*}
        \ell-m 
        &>
        (m+1)(b-1)
        \\
        \ell - m 
        & > m(b-1) + (b-1) 
        \\
        \ell 
        &> bm + (b-1)
        \\
        \frac{\ell - (b -1)}{b} 
        &> m
        \\
        \ell\of*{\frac{1}{b} - \frac{b-1}{b\ell}}
        &> m 
    \end{align*}
    Providided $\ell$ is sufficiently large, then 
    this is true for all terms in our sum.

    Thus we can bound all the terms in our sum by the last term,
    since all of the terms are increasing.
    This gives us 
    \[
        b\sum_{m<\ell\of*{\frac{1}{b}-\epsilon}}
        \binom{\ell}{m} (b-1)^{\ell-m}
        \le b\ell^3 \of*{\frac{1}{b}-\epsilon}
        \cdot \binom{\ell}{\ell \of*{\frac{1}{b}-\epsilon}}
        (b-1)^{\ell - \ell \of*{\frac{1}{b}-\epsilon}}
    \]
    The $\ell^3$ comes from replacing 
    $\ell\of*{\frac{1}{b}-\epsilon}$ inside the binomial.
    \[\le 
    b\ell^3 \binom{\ell}{\of*{\frac{1}{b}-\epsilon}\ell}
    (b-1)^{\ell\of*{\frac{b-1}{b} +\epsilon}}
    \]
    \[
        =O\of*{
            \ell^{5/2}
            \braks*{
                \of*{
                    \frac{b}{1-b\epsilon}
                }^{
                    \frac{1}{b}-\epsilon
                }
                \of*{
                    \frac{b}{
                        b-1+b\epsilon
                    }
                }^{
                    \frac{b-1}{b} + \epsilon
                }
                \of*{
                    b-1
                }^{
                    \frac{b-1}{b} + \epsilon
                }
            }^\ell
        }
    \]
    Doing some algebra, we get
    \[
        =O\of*{
            \ell^{5/2}
            b^\ell
            \braks*{
                \of*{
                    1-b\epsilon
                }^{
                    -\of*{\frac{1}{b}-\epsilon}
                }
                \of*{
                    1+
                    \frac{b}{
                        b-1
                    }
                    \epsilon
                }^{
                    -\of*{\frac{b-1}{b} + \epsilon}
                }
            }^\ell
        }
    \]
    I claim it suffices to show what's in brackets 
    is $b^{-2\delta'}$, because 
    \[ \ell^{5/2} = O(b^{\delta' \ell}),
    \]
    which will give the first sum being bounded 
    by $O\of*{b^{\ell(1-\delta')}}$.
    So consider 
    \[\of*{1-b\epsilon}^{
        -\of*{
            \frac{1}{b}-\epsilon
        }
    }
    \of*{
        1+\frac{b}{b-1}\epsilon
    }^{
        -\of*{\frac{b-1}{b} + \epsilon}
    }
    \]
    \[
    = \exp\of*{
        -\of*{
            \frac{1}{b}-\epsilon
        }
        \log\of*{
            1-b\epsilon
        }
        -
        \of*{
            \frac{b-1}{b} + \epsilon
        }
        \log \of*{
            1+\frac{b}{b-1}\epsilon
        }
    }
    \]
    Then we can apply the Taylor series for $\log (1+x)$,
    which gives $\log(1+x) =x +\frac{x^2}{2}+\frac{x^3}{3}
    =x+O(x^2)$.
    Applying it to our case, we get
    \[\exp\of*{
        -\frac{b^2}{2(b-1)} \epsilon^2 
        +O(\epsilon^3)
    }
    \]
    So if $\epsilon$ is small enough, this is less 
    than $1$, so we can find $\delta'$.

    This proves the $k=1$ case.

    For the general case, we have a trick!
    $(\epsilon,k)$-normality for base-$b$ 
    acts like $(\epsilon,1)$-normality for 
    base-$b^k$.

    For example, consider 
    $1205287$, for $(\epsilon,3)$-normality, 
    then we would really look at 
    $(120)(528)*$ and $*(205)(287)$, and 
    $**(052)**$. There are three shifts we need to
    care about.

    If a string of base-$b$ digits is \emph{not}
    $(\epsilon,k)$-normal, then at least one of 
    these base $b^k$ ``shifts'' is not 
    $(\epsilon,1)$-normal. 
    (Perhaps ($\epsilon/k$,1)-normal? Exercise.)

    So the number of non-$(\epsilon,k)$-normal 
    base-$b$ strings of length $\ell$ is 
    $\le$ the number of non-$(\epsilon,1)$-normal
    base $b^k$ strings of length 
    $\floor*{\frac{\ell}{k}}$ or $\floor*{\frac{\ell}{k}}-1$
    times $k$.

    This equals 
    \[ k O\of*{
        \of*{
            b^k
        }^{
            \floor*{
                \frac{\ell}{k}
            }(1-\delta')
        }
        +
        \of*{
            b^k
        }^{
            \floor*{
                \frac{\ell}{k}
                -1
            }(1-\delta')
        }
    }
    = O\of*{b^{\ell(1-\delta')}}.
    \]

\end{proof}

\section{The techniques in the proof}

Remember the sum 
\[\sum \binom{\ell}{m} (b-1)^{\ell-m}? \]
Often functions increase for a while, reach a 
maximum, and then decrease. It's often
easy to bound the sums on the tails.

Sometimes you can make it look like 
\[e^{-\lambda n^2},\]
Then you can say that the sum is like the integral,
and that this integral is easy. (This is one place
we get $\sqrt{\pi}$s popping out).

Another useful trick is 
\[(1+a)^b = e^{b\log (1+a)} \approx e^{ab} \]
if $a$ is small.

\begin{proof}[Normal number construction theorem]
    We want to show $0.f(1)f(2)f(3)\cdots$ is base-$b$
    normal. We must show that for any string, $s$,
    \[\lim_{N\to\infty} \frac{\nu_s(x,N)}{N} 
    = \frac{1}{b^{|s|}},\]
    where $\nu_s(x,N)$ counts the number of appearances
    of $s$ in the first $N$ digits of $x$.

    For a given $N$, let $m(N)$ be such that the $N$th
    digit of $x$ lies in $f(m)$. In particular,
    \[
        \sum_{n=1}^{m-1} \nu(f(n)) 
        < N 
        \le 
        \sum_{n=1}^m \nu(f(n)).
    \]
    By our assumptions 
    \[
        \nu(f(m)) = o\of*{
            \sum_{n=1}^m
            \nu(f(n))
        }.
    \]
    So therefore,
    \[
        N\sim 
        \sum_{n=1}^m \nu(f(n)).
    \]
    Also $m=o(N)$, because of our assumption that 
    the lengths of $\nu(f(n))$ are mostly large.
    And 
    \[
        \nu(f(m)) = o(N).
    \]
    Therefore 
    \[
        \nu_s(x,N) = 
        \nu_s(f(1)f(2)\cdots f(m)) + O(\nu(f(m)))
        = \nu_s(f(1)f(2)\cdots f(m)) + o(N).
    \]


\end{proof}

\end{document}