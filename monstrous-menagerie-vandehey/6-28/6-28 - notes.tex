\documentclass{article}

\title{Monstrous Menagerie with Vandehey 6/28}
\author{Jason Schuchardt}

\usepackage{jragonfyre}
\newcommand\cft[1]{\frac{1}{#1+}}
\newcommand\hoint{[0,1)}
\newcommand\recip[1]{\frac{1}{#1}}

\begin{document}
\section{Computing CF Expansions}

Consider $\sqrt{2}$.

We know
\[\sqrt{2} = 1 + \cft{a_1} \cft{a_2}\cdots \]
so
\[ \sqrt{2}-1 = \frac{a_1+\theta_1},\]
and
\[ \frac{1}{\sqrt{2}-1} = a_1+\theta_1, \]
so $a_1 = \floor*{\frac{1}{\sqrt{2}-1}}$.

Then \[\sqrt{2} = 1 + \cft{a_1}\frac{1}{a_2+\theta_2},\]
and
\[ \frac{1}{\sqrt{2}-1} = a_1 + \frac{1}{a_2+\theta_2}, \]
so 
\[\frac{1}{\frac{1}{\sqrt{2}-1}-a_1} = a_2 + \theta_2.\]
Thus
\[ a_2 = \floor*{\frac{1}{\frac{1}{\sqrt{2}-1}-a_1}}\]

\subsection{Comparing to base-10}

Consider $x\in[0,1)$.
$a_1$ corresponds to $d$ where $x\in [\frac{d}{10},\frac{d+1}{10})$.
$a_2$ corresponds to $d$ where
$x\in [\frac{a_1}{10}+\frac{d}{100}, \frac{a_1}{10}+\frac{d+1}{100})$

Consider instead $x_1=10x-a_1$, then $a_2=d$, where
$x_1\in [\frac{d}{10},\frac{d+1}{10})$,
$x_2 = 10x-a_2 = \set{10x}$.
$a_3=d$ where $x_2 \in [\frac{d}{10},\frac{d+1}{10})$ and so on.

\section{Dynamical system (fibred system)}

\begin{definition}
A \emph{dynamical system} consists of 
\begin{enumerate}[1)]
    \item A space $X$, e.g., $[0,1)$,
    \item A set of digits $\DDD$, finite or countably infinite,
    \item A partition $\XXX$ of $X$ into 
        \[ \set{X_d : d\in \DDD},\]
    \item And a transformation, $T:X\to X$, which is injective
        on each $X_d$,
    \item A $\sigma$-algebra $\calF$ and a measure $\mu$.
\end{enumerate}
\end{definition}

Given $x\in X$, let $a_1(x) = d$, where $x\in X_d$.
Let $a_2(x)=a_1(Tx)$. In general, let
$a_{n}(x) = a_1(T^{n-1}x)$.

\begin{definition}
    The sequence $(a_1(x),a_2(x),a_3(x),\ldots,a_n(x),\ldots)$
    is the expansion of $x$ in this system.
\end{definition}

\begin{example}
    For base-$b$, we have
        $X=[0,1)$,
        $\DDD = \set{0,1,\ldots,b-1}$,
        $X_d = [\frac{d}{b},\frac{d+1}{b})$,
        and
        $Tx = \set{bx}$.

    Picture: Graph the transformation for base 2, base 3.
\end{example}

\begin{example}
    $\beta$-expansion is not great. We'll use \emph{greedy}
    $\beta$-expansion, which tries to maximize leftward digits.
    With greedy $\beta$-expansion, we have $X=[0,1)$,
    $\DDD=\set{0,1,\ldots,\ceil{\beta}-1}$,
    \[ X_d = 
    \begin{dcases} 
        \left[\frac{d}{\beta},\frac{d+1}{\beta} \right) & d < \ceil{\beta}-1\\
        \left[\frac{d}{\beta}, 1 \right) & d = \ceil{\beta}-1,
    \end{dcases}
    \]
    and $Tx = \set{\beta x}$

    Picture: Graph the transformation.

    Note $\set{b\set{bx}} = \set{b^2x}$, when $b\in\ZZ$, but
    not when $b=\beta$ is not an integer.

    Does this do what we want it to do?

    We have
    \[ x = \sum_{i=1}^\infty \frac{a_i}{\beta^i}.\]
    Then
    \[ \set{\beta x} = \set*{a_1+\sum_{i=1}^\infty \frac{a_{i+1}}{\beta}}.\]
    It's not clear that the fractional part just deletes the $a_1$,
    but if we're using greedy expansions, this should be true.
\end{example}

\begin{example}
    For regular CF, $X=[0,1)$, $\DDD = \NN\cup\set{\infty}$,
    \[ X_d = 
    \begin{dcases}
        \left(\frac{1}{d+1},\frac{1}{d}\right] & d \inN\\
        \set{0} & d = \infty
    \end{dcases},
    \]
    and 
    $Tx = \set*{\frac{1}{x}}$.

    Checking that this works, we have
    \[ T\cft{a_1}\cft{a_2}\cft{a_3}\cdots = 
    \set*{a_1+\cft{a_2}\cft{a_3}\cdots}
    = \cft{a_2}\cft{a_3}\cft{a_4}\cdots
    \]

    What's with the $\infty$?
    $T(1/5) = \set{5} = 0$.

    Picture: Plot $Tx$.

    Even CF picture: Reflects $1/x$ back and forth between
    $y=0$ and $y=1$.

    Odd CF: Flips the intervals that the even one does not.
\end{example}

\begin{example}
    For L\"uroth series, we have
    $X=(0,1]$, $\DDD = \NN_{\ge 2}$,
    $X_d = \left(\frac{1}{d},\frac{1}{d-1}\right]$,
    $Tx = \set{d(d-1)x}$, for $x\in X_d$.

    Picture: Lines of increasing slope.

    This is why L\"uroth series have the same periodic points
    as base-$b$ expansions. We keep multiplying by an integer,
    and subtracting an integer, so we stay rational, and
    maybe shrink the denominator.
\end{example}

The injectivity of $T$ on $X_d$ is necessary for the
uniqueness of representations, but not necessarily sufficient.

\section{Convergence}



\end{document}