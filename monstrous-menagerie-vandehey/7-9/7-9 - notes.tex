\documentclass{article}

\title{Monstrous Menagerie with Vandehey 7/9}
\author{Jason Schuchardt}
\date{\today}

\usepackage{jragonfyre}
\newcommand\cft[1]{\frac{1}{#1+}}
\newcommand\hoint{[0,1)}
\newcommand\recip[1]{\frac{1}{#1}}
\newcommand\cylset[1]{C_{[{#1}]}}


\begin{document}
\maketitle

\section{Applications of the Ergodic Theorem}

Last time we talked about the pointwise ergodic theorem, now 
we'll see some applications of it.

Let's suppose $(X,T,\mu)$ is an ergodic system with probability
measure.
Consider $f=1_A$.
Since 
\[ \int f\,d\mu = \mu(A), \]
therefore
\[\int \abs{f}\, d\mu = \int f \, d\mu = \mu(A)<\infty.\]
Thus $f\in L^1(X,\mu)$ for any $A\in\calA$.

We can therefore apply the pointwise ergodic theorem.
Namely for almost all $x\in X$, we have 
\[\lim_{N\to\infty} \frac{1}{N} \sum_{i=0}^{N-1} f(Tx) 
=\int f\, d\mu = \mu(A).\]
Thus for an ergodic system, almost all points spend $\mu(A)$ 
fraction of the time on average in $A$.

Can consider $A=C_s$ for fibred systems. In words, for almost all
$x$, the limiting frequency that $s$ appears in the expansion 
of $x$ should be the measure of the corresponding cylinder 
set, $C_s$.

\begin{example}
    For example, with base-$b$ expansions, 
    \[\lambda(C_{[a_1,\ldots,a_k]}) = b^{-k}. \]
\end{example}

Returning to general fibred systems,
for each $s$, let $B_s$ denote the set of $x$ where this 
\emph{doesn't} happen.
Note that $\mu(B_s)=0$, so 
\[\mu\of*{\bigcup_{s} B_s} = 0,\]
since there are only countably many finite strings.

\begin{definition}
    A point $x\in X$ is \emph{$T$-normal} if for every admissible
    string $s$ we have that 
    \[\lim_{N\to \infty} \frac{1}{N} \sum_{i=0}^{N-1} 1_{C_s}(T^ix)
    = \mu(C_s).\]
\end{definition}

\begin{theorem}[Borel Normal Number Theorem]
    Almost all $x\in X$ are $T$-normal. 
\end{theorem}

Almost all is doing a lot of work here. A system for which every 
point is normal, is called \emph{uniquely ergodic}. For example,
an irrational rotation of the circle is uniquely ergodic.
However, this often fails. In particular,
if there are any periodic points, this
will almost always fail to be true.

\begin{proposition}
    Almost all (Lebesgue) $x\in [0,1)$ are \emph{absolutely normal},
    i.e., normal in every base $b\ge 2$ simultaneously.
\end{proposition}

This proposition follows immediately from the Borel Normal 
Number Theorem, since there are only countably many bases $b$.

Side note: There is also the notion of a number looking random.
Our condition says that the expectation of a base string appearing
is what it should be. What about the variance? If the variances
are what they should be, then the number ``looks random.''
It is also true that almost every number looks random, but 
ergodicity doesn't cover it, so we won't go into it.

\begin{proposition}
    If $x$ is base-$b$ normal, then it is base $b^r$ normal for 
    any rational $r$ such that $b^r$ is an integer.
\end{proposition}

This should not be surprising. For example, for bases 2 and 4,
we obtain base 2 from expanding the digits in base 4, 
and obtain base 4 from grouping the digits in base 2.

\begin{proposition}
    Suppose $b\ne c^r$ for any rational $r$. Then the set of 
    base-$b$ but not base-$c$ normal numbers is uncountable.
\end{proposition}

In fact, this set has positive Hausdorff dimension.

This was first proved for base-2, base-3, by proving that
almost all of the points in the middle thirds cantor set 
are base-2 normal.

What if we throw in CF expansions?

\begin{proposition}[Due to Vandehey]
    \emph{Assuming GRH},
    there exists $x\in [0,1)$ that is RCF-normal, but not base-
    $b$ normal for any $b\ge 2$.
\end{proposition}

So far they haven't been able to find a proof that doesn't rely on
the Riemann hypothesis.

If we ask for a particular number, $\pi$, or $e$, the question 
of whether it is normal in a particular base or CF expansion 
is either obvious (like $e$ in RCF), or entirely intractable
with our current methods.

It is hard to construct normal numbers. Alan Turing worked on it.
\begin{theorem}[Champernowne, 1922]
    \[ 0.123456789101112131415161718192021222324\ldots \]
    is base-10 normal.
\end{theorem}
Champernowne published as an undergrad, then moved on to economics.

\begin{theorem}[Copeland-Erod\H{o}s]
    \[0.23571113171923\ldots \]
    is base-10 normal.
\end{theorem}

These theorems generalize in the sense that we can apply these 
constructions by writing the integers or primes in order 
in other bases to obtain normal numbers in other bases.

\begin{definition}
\[ \omega(n)=\#\text{ of distinct prime factors of $n$}\]
\end{definition}

For example, $\omega(12)=2$.

Let $\omega_y(n)$ be the string of digits of the last $y$
portion of $\omega(n)$.
For example, if $\omega(N) = 5290$, then $\omega_{1/2}(N) = 90$.
Then we have the theorem
\begin{theorem}[Vandehey]
    \[0.\omega_y(1)\omega_y(2)\omega_y(3)\omega_y(4)\ldots \]
    is normal if and only if $0<y\le 1/2$.
\end{theorem}

The reason for the $y \le 1/2$ is that $\omega(N)\approx \log\log N$,
and $\log\log$ changes its digits so slowly that something 
will dominate, preventing normality.

The first $9$ appears at digit $2\cdot 3 \cdot 5\cdot 7 \cdot 11 
\cdot 13\cdot 17\cdot 19 \cdot 23$, since that is 
the first number for which $\omega(n) = 9$.

The goal for this week is to show you that Champernowne's theorem 
is true. Before we get there, we'll show that some things are 
ergodic.

\section{How to prove that a system is ergodic?}

\begin{lemma}[Knopp's Lemma]
    Suppose $(X,\mu)$ is a probability space. Suppose that 
    $\AAA$ is a semi-algebra that generates $\calA$. 
    Let $E\in \calA$. If there 
    exists $\delta > 0$ such that for all $A\in\AAA$,
    \[\mu(E\cap A) \ge \delta \mu(A)\]
    then $\mu(E)=1$
\end{lemma}


\end{document}