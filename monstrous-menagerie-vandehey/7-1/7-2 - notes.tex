\documentclass{article}

\title{Monstrous Menagerie with Vandehey 7/2}
\author{Jason Schuchardt}

\usepackage{jragonfyre}
\newcommand\cft[1]{\frac{1}{#1+}}
\newcommand\hoint{[0,1)}
\newcommand\recip[1]{\frac{1}{#1}}
\newcommand\cylset[1]{C_{[{#1}]}}
\newcommand\diam{\operatorname{diam}}

\begin{document}
\maketitle

\section{Probability and Invariant measures.}

\begin{definition}
    $\mu$ is a \emph{probability measure} if $\mu(X)=1$.
\end{definition}

If $\mu(X)$ is finite, we can always renormalize to get a
probability measure:
\[ \mu^*(A) = \frac{\mu(A)}{\mu(X)}. \]

Sometimes $\mu(X)$ is infinite. These are basically the only two
possibilities (other than the zero measure, which is boring).

\begin{definition}
Given $(X,\calA,T,\mu)$ 
(space, $\sigma$-algebra, transformation, measure)
we say $\mu$ is \emph{$T$-invariant,} if 
for all $A\in \calA$, 
\[\mu(T\inv A)=\mu(A),\]
where $T\inv(A) = \set{x\in X : Tx\in A}$.
\end{definition}

Two questions here: Why is it important? Because almost 
everything we want to do requires invariance.

If $\mu$ is not invariant,
define $\mu_k(A) = \mu(T^{-k}A)$,
and then we can take a sort of limit of $\mu_k$ to get an
invariant measure.

The other question is: Why is it $T\inv A$, why not just
$TA$? Because $T\inv$ preserves all of the information.
I can start with two points $x$ and $y$ and apply $T$ and
get a single point. For example with base $b$ expansions,
two points which differ in their first digit end up at the 
same point after applying $T$.

On the other hand, for $T\inv$ we know where we came from,
we can just apply $T$ to any point in $T\inv A$.
$TT\inv x = x$, but $T\inv T x = ?$.

\subsection{Proving invariance}

\begin{theorem}
    Suppose $\AAA$ is a semi-algebra that generates a $\sigma$-
    algebra $\calA$. If $\mu(T\inv A) = \mu(A)$ for all
    $A\in\AAA$, then $\mu$ is $T$-invariant.
\end{theorem}

\begin{proof}
    Define $\mu'(A) = \mu(T\inv A)$. $\mu'$ is also a measure
    on $(X,\calA)$, so by the uniqueness of the Caratheodory
    Extension Theorem, and the fact that the given information
    is that $\mu'(A)=\mu(A)$ for $A\in\AAA$, we conclude 
    that $\mu=\mu'$ as desired.
\end{proof}

\begin{example}
    For base $b$, $\lambda$ is $T$-invariant.
    \[ T\inv x = \set*{\frac{0}{b} + \frac{x}{b},
    \frac{1}{b} + \frac{x}{b}, \ldots, 
    \frac{b-1}{b} + \frac{x}{b}}\]
    \[ \lambda(T\inv [x,y]) = 
    \lambda \of*{\bigcup_{i=0}^{b-1} \braks*{\frac{i+x}{b}, \frac{i+y}{b}}
    }
    = \sum_{i=0}^{b-1} \lambda\of*{\braks*{\frac{i+x}{b},\frac{i+y}{b}}}
    = \sum_{i=0}^{b-1} \frac{y-x}{b} = y-x
    \]

    Since the semi-algebra of intervals generates the 
    Lebesgue $\sigma$-algebra, $\lambda$ is $T$-invariant.
\end{example}

\begin{exercise}
    Prove that $\lambda$ is not invariant for regular
    CF expansions.
\end{exercise}

\end{document}