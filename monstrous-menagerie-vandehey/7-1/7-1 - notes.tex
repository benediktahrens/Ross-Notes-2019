\documentclass{article}

\title{Monstrous Menagerie with Vandehey 7/1}
\author{Jason Schuchardt}

\usepackage{jragonfyre}
\newcommand\cft[1]{\frac{1}{#1+}}
\newcommand\hoint{[0,1)}
\newcommand\recip[1]{\frac{1}{#1}}
\newcommand\cylset[1]{C_{[{#1}]}}
\newcommand\diam{\operatorname{diam}}

\begin{document}

\section{Convergence of Expansions}

\begin{theorem}
    Suppose either 
    \begin{enumerate}
        \item $X$ is a metric space of finite diameter, and for any $X_d$ and all $x,y\in X_d$; there is
            a uniform constant $\epsilon$ such that
            \[ \frac{d(Tx,Ty)}{d(x,y)} \ge 1+\epsilon \]
        \item $X\subseteq \RR$, a bounded subset, each $X_d$ is an interval, $T$ is continuous on $X_d$,
            differentiable on interior of $X_d$ and there exists $\epsilon > 0$ such that 
            $|T'x| \ge 1+\epsilon$ for all $x$ where the derivative exists.
    \end{enumerate}

    Then the expansion converges for all points in $X$.
\end{theorem}

\begin{example}
    Hurwitz complex CFs 
    \[ X = \set{x+iy : x,y \in [-1/2, 1/2)},\]
    \[ Tz = \set*{\frac{1}{z}}\]

    Then 
    \[ d\of*{\frac{1}{z}, \frac{1}{z'}} = \frac{d(z,z')}{|z||z'|}\]
\end{example}

\begin{proof}
    Note case 1 implies case 2 by the mean value theorem.

    For case 1, consider a cylinder set \[C_{[a_1]} = X_{a_1},\]
    we know that for all $x,y\in C_{[a_1]}$,
    \[ \frac{d(Tx,Ty)}{d(x,y)} \ge 1 +\epsilon. \]

    Consider $\cylset{a_1,a_2}$. 
    Then if $x\in \cylset{a_1,a_2}$, $Tx\in \cylset{a_2}$.

    So for all $x,y\in \cylset{a_1,a_2}$, then 
    \[ \frac{d(T^2x,T^2y)}{d(x,y)} = \frac{d(T^2x,T^2y)}{d(Tx,Ty)} \cdot 
    \frac{d(Tx,Ty)}{d(x,y)} \]
    \[ \ge (1+\epsilon)^2. \]

    By induction, for all $x,y\in \cylset{a_1,\ldots,a_k}$,
    \[ \frac{d(T^kx,T^ky)}{d(x,y)} \ge (1+\epsilon)^k.\]

    Since our metric space has finite diameter, 
    $d(\cdot, \cdot) \le M$ for all inputs (for some $M$). 
    Thus for all $x,y\in \cylset{a_1,\ldots,a_k}$, 
    \[ d(x,y) \le \frac{M}{(1+\epsilon)^k} \]

    So as $k\to \infty$, 
    \[ \diam(\cylset{a_1,\ldots,a_k}) \to 0.\]
    This completes the proof.
\end{proof}

Note that this doesn't work for regular continued fractions, since 
we have $Tx=\set*{\frac{1}{x}}$, which has derivative $-1$
at $1$.

How do we know the regular continued fractions do converge?
There's only a problem at $x=1$, and near $1$ we end up near $0$.
We can thus do two steps, since things will expand after 
a possibly bad step.

Question: Where does a cylinder set get its name? Not sure.
There's lots of weird terminology.

\section{Measures}

Bergelson: The more you talk about it, the less I need to!

\begin{definition}
    A \emph{measure} is a way to measure the size of sets. (It will
    not usually measure all sets).

    A measure $\mu : \calB \to \RR_{\ge 0}\cup\set{\infty}$,
    $\calB\subseteq \powerset{X}$,
    satisfies
    \begin{enumerate}
        \item $\mu(A)\ge 0$ for all $A\in\calB$.
        \item $\mu(\nullset)=0$.
        \item If $A_1,A_2,\ldots,A_n,\ldots$ are all disjoint
            \[ \mu\of*{\bigcup_{i=1}^\infty A_i }
            = \sum_{i=1}^\infty \mu(A_i)\]
    \end{enumerate}

    Why the $\calB$? Well, it's not possible to always 
    consistently assign measures to sets when we take all of
    the subsets, so we restrict ourselves to certain sets
    that we care about. In particular, $\calB$, is a 
    \emph{$\sigma$-algebra.} 
    A $\sigma$-algebra, $\calA$, on a set $X$ must satisfy
    \begin{enumerate}
        \item $\calA\subseteq \powerset{X}$,
        \item $X\in \calA$,
        \item if $A\in\calA$, then $X\setminus A\in\calA$,
        \item and if $A_1,A_2,\ldots,A_n,\ldots \in \calA$,
            then \[ \bigcup_{i=1}^\infty A_i \in \calA. \]
    \end{enumerate}

    Measures are always defined on families of sets that
    form $\sigma$-algebras.
\end{definition}

Let $\AAA$ be any collection of subsets of $X$. Then there is a 
smallest $\sigma$-algebra containing $\AAA$, $\sigma(\AAA)$. Note. If 
$\calF_1,\calF_2$ are $\sigma$-algebras, then 
$\calF_1\cap \calF_2$ is also a $\sigma$-algebra. This is also
true for arbitrary intersections of $\sigma$-algebras.

Thus we have that
\[ \sigma(\AAA) = \bigcap_{\calA \supseteq \AAA} \calA, \]
where the $\calA$ are all $\sigma$-algebras.

\begin{example}
    ``Obvious choice of measure on $[0,1]$''
    (This will also work for $\RR$, $\RR^n$.)

    For intervals, we should have $\lambda([a,b]) := b-a$.

    If we let $\AAA$ be the collection of all subintervals of
    $[0,1]$, $\sigma(\AAA)$ is called the \emph{Borel} $\sigma$-
    algebra on $[0,1]$, $\calB([0,1])$.
    We will see $\lambda$ extends to $\calB([0,1])$.
    Since $\lambda$ is defined on $\calB([0,1])$, we say 
    that it is a Borel measure.
\end{example}

\section{Semi-algebras}
% why are we talking about semi-algebras?

\begin{definition}
    A collection $\AAA$ of subsets of $X$ is a semi-algebra if 
    \begin{enumerate}
        \item $X,\nullset\in\AAA$,
        \item If $A,B\in\AAA$, then $A\cap B\in \AAA$.
        \item If $A,B\in \AAA$, then $A\setminus B$ must be a
            union of a finite number of disjoint sets in 
            $\AAA$.
    \end{enumerate}

    Note that (3) implies $\nullset\in \AAA$, 
    since $X\setminus X =\nullset$.
\end{definition}

If $\AAA$ is a semi-algebra, and $\mu$ is a ``measure'' on 
$\AAA$, then the measure $\mu$ extends to a unique measure
$\mu^*$ on $\sigma(\AAA)$. % Aha! Caratheodory extension theorem?
(This is Caratheodory's Extension Theorem).

\begin{example}
    If $\DDD$ is finite, then the collection
    of cylinder sets (and $X$ and $\varnothing$)
    is a semi-algebra.

    Let's verify the rules.
    \begin{enumerate}
        \item By definition, this is satisfied.
        \item We'll ignore $X$ and $\nullset$.
            Consider $C_s,C_t\in \DDD$.
            If $s=[a_1,a_2,\ldots, a_k]$, 
            $t=[b_1,b_2,\ldots,b_\ell]$, $k\le \ell$.
            If $a_i\ne b_i$ for some $1\le i \le k$, then
            $C_s\cap C_t = \nullset$. Otherwise
            $a_i=b_i$ for $1\le i\le k$, so $C_s\cap C_t=C_t$.
        \item If $C_s\cap C_t = \varnothing$,
        then $C_s \setminus C_t = C_s$. Otherwise,
        if $C_s\cap C_t = C_t$, 
        then $C_t\setminus C_s = \varnothing$, so we may assume we
        are considering $C_s\setminus C_t$, which is the set of all
        $x\in X$ whose first $k$ digits are $s$, but whose
        first $\ell$ digits are not $t$.

        Since $\DDD$ is finite, there are a finite number of strings
        of length $\ell$ whose first $k$ digits are $s$, but are 
        not equal to $t$. Then $C_s\setminus C_t$ will be the 
        disjoint union of the finitely many corresponding 
        cylinder sets.
    \end{enumerate}

    What if our digit set $\DDD$ is infinite?
    We can use ``extended cylinder sets,'' 
    \[ C^*_{[a_1,\ldots,a_k]} =
    \bigcup_{d\ge a_k} C_{[a_1,\ldots,a_{k-1},d]}.\]
    This is a sort of compactification argument.

    This is how we get a nice semi-algebra for continued 
    fractions for example.
\end{example}

\subsection{Completing a $\sigma$-algebra and measure}

Those familiar with measure theory will know that we don't usually
work with the Borel $\sigma$-algebra or Borel measures. We need
to complete the measure.

We have a set $X$, a $\sigma$-algebra $\calB$, and a measure $\mu$.
We want to add all subsets of zero-measure sets to our 
$\sigma$-algebra and fill everything in accordingly.

When you complete the Borel $\sigma$-algebra and measure, 
you get the Lebesgue $\sigma$-algebra and measure, which we
will also denote by $\lambda$.

\subsection{Terminology}

\begin{definition}
    A function $f: X\to Y$ is \emph{measurable} with respect to 
    $\sigma$-algebras $\calB$ on $Y$ and $\calA$ on $X$ if 
    for all $B\in \calB$, $f^{-1}(B) \in \calA$.

    Recall $f\inv(A) = \set{x\in X: f(x) \in A}$.
\end{definition}

\begin{definition}
    A set, $A$, has \emph{zero} or \emph{null measure} if 
    $\mu(A)=0$. $A$ has \emph{full measure} if 
    $\mu(A)=\mu(X)$ (assuming $\mu(X) < \infty$).
\end{definition}

\begin{definition}
    Something happens for \emph{almost all points in $A$},
    or \emph{almost everywhere in $A$}, if the set of points
    $x\in A$ where the thing doesn't happen has measure zero.

    For example, $\lambda(\QQ\cap [0,1]) = 0$, so almost every 
    real number is irrational. 
\end{definition}

\section{Integration of functions with respect to measures}


We know about integration of functions on intervals in $\RR$:
\[ \int_a^b f(x)\,dx \text{ ``signed area under curve''}\]

What then is 
\[\int_A f\,d\mu? \]

First, if \[ f=1_A =
\begin{dcases} 1 & x\in A \\ 0&x\not\in A, \end{dcases} \]
then
\[ \int_X 1_A\,d\mu = \mu(A).\]

Second, if $f= \sum_{i=1}^k c_i 1_{A_i}$, where the $A_i$ are 
disjoint, we have
\[ \int_{X} f\,d\mu 
= \sum_{i=1}^k c_i \int_X 1_{A_i}\,d\mu
= \sum_{i=1}^k c_i\mu(A_i).\]

Third, what if $f$ can be expressed as 
\[ f_k \to f,\]
where $f_k$ are as in the previous step?

Then 
\[ \int_X f\,d\mu = \lim_{k\to\infty} \int_X f_k\,d\mu. \]

Finally,
\[ \int_A f\,d\mu = \int_X f\cdot 1_A \, d\mu.\]

\begin{remark}
    Most of the time, 
    \[ \int_X f\,d\lambda = \int f\, dx. \]

    Some functions can only be integrated in the world of measure,
    for example $1_{\QQ\cap [0,1]}$. This is because the Riemann
    integral won't converge.

    If $\mu(A)=0$, whenever $\lambda(A)=0$, then there exists 
    a function $h$, such that 
    \[\int_Xf\,d\mu = \int_X f\cdot h \, d\lambda. \]
    $h$ is the Radon-Nikodym derivative of $\mu$ with respect
    to $\lambda$.
\end{remark}

For example, there is a nice measure, the Gauss measure, $\mu$, and we get 
\[d\mu = \frac{1}{(\log 2) (1+x)} d\lambda\]


\end{document}