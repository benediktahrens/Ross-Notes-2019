\documentclass{article}

\usepackage{jragonfyre}
\usepackage{hyperref}

\title{Type Theory Set - 1}
\author{Jason Schuchardt}

\begin{document}

\maketitle

Type theory is built on many years of work from areas of math
and computer science
that aren't usually studied in high school or the typical
undergraduate math curriculum. In particular, the simply typed 
lambda calculus is built on the notation and ideas of the
(untyped) lambda calculus.
(\url{https://en.wikipedia.org/wiki/Lambda_calculus})

\begin{definition}
    We should begin by defining the untyped lambda calculus.
    The untyped lambda calculus is essentially a syntactic game,
    i.e., we form expressions according to certain rules,
    and manipulate (evaluate) them according to other rules.

    Formation rules. A lambda expression is recursively defined,
    and consists of one of the following things:
    \begin{enumerate}
        \item A variable, denoted by a lowercase latin letter,
            with an optional subscript, e.g.
            \[a,x_1,x_2,v,y,w\]
        \item A function definition: if $x$ is any variable, and $L$ is 
            a lambda expression, then 
            \[\lambda x.L\]
            is a lambda expression, which we think of as a
            function which takes a variable $x$, and is defined 
            by the expression $L$. This is also called 
            $\lambda$ abstraction.
        \item Function application. If $L$ and $M$ are lambda
            expressions, then \[ (L\ M) \] is a lambda expression
            that we think of as representing the application
            of the function defined by the expression $L$ to 
            the expression $M$.
    \end{enumerate}

    Evaluation rules. Lambda expressions may be evaluated 
\end{definition}

\begin{enumerate}[(1)]
    \item 
\end{enumerate}

\end{document}