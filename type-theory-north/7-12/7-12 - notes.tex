
\documentclass{article}


\title{Type Theory with Paige North 7/10}
\author{Jason Schuchardt}

\usepackage{jragonfyre}
\usepackage{turnstile}
\usepackage{bussproofs}
\usepackage{booktabs}
\newcommand\blank{ \underline{\phantom{$\quad\quad$}} }
\newcommand\TYPE{\text{ \scshape{type}}}
\newcommand\ctxt{\text{ ctxt}}
\newcommand\add{\operatorname{add}}
\newcommand\mult{\operatorname{mult}}
\newcommand\List{\textsc{List}}
\newcommand\nil{\textrm{nil}}
\newcommand\con{\operatorname{con}}
\newcommand\pred[1]{\operatorname{#1}}
\newcommand\IdT{\mathrm{Id}}
\newcommand\refl{\mathrm{refl}}
\newcommand\BB{\mathbb{B}}
\newcommand\Rel{\operatorname{Rel}}

\begin{document}

\maketitle

\section{Product Types}

Generalize the function type.

\subsection{In Sets:}

For a one element set $1=\set{*}$, and a set $X$
we have 
\[\set{\text{elements of $X$}}
\longleftrightarrow 
\set{\text{functions $1=\set{*}\to X$}}
\longleftrightarrow 
X
\longleftrightarrow
\prod_{x\in \set{*}} X.
\]
For a two element set, $2$, and a set $X$,
we have
\[\set{\text{ordered pairs of $X$}}
\longleftrightarrow 
\set{\text{functions $2\to X$}}
\longleftrightarrow 
X\times X
\longleftrightarrow
\prod_{x\in 2} X.
\]
Then 
\[\set{\text{sequences in $X$}}
\longleftrightarrow 
\set{\text{functions $\NN\to X$}}
\longleftrightarrow 
X^{\NN}
\longleftrightarrow
\prod_{x\in \NN} X.
\]

One way to think of functions is 
as tuples. For functions $A\to X$, we 
can think of these as tuples in $X$ whose entries are labeled 
elements of $A$.

If we have an indexed family $E(b)$ over $B$, then we can form
\[\prod_{b\in B} E(b), \]
the set of generalized tuples $x$, where $x_b \in E(b)$.

This is in bijection with sections of the map
\[\pi:\coprod_{b\in B} E(b) \to B.\]
A \emph{section} of the map $\pi$ is a map 
\[ s:B\to \coprod_{b\in B}E(b) \]
such that $\pi s = 1_B$. In other words, $s(b)\in \pi\inv(b)$
for all $b$.

This is the concept that we want to generalize in type theory.
If $E$ is trivially indexed by $B$, then 
\[ \prod_{b\in B} E
\simeq \set*{\text{sections of $\coprod_{b\in B}E\simeq B\times E \to B$}}
\simeq \Hom(B,E). \]

This is the sense in which we are generalizing functions.

\section{$\Pi$-types}

We start with the rules for $\Pi$-types.
\begin{enumerate}
    \item $\Pi$-formation 
        \[ \frac{
            \Gamma \yields B: U
            \qquad \Gamma,x:B\yields E(x):U
        }{
            \Gamma \yields \prod_{x:B} E(x) : U
        }\]
    \item $\Pi$-introduction
        \[\frac{\Gamma,x:B\yields e(x) :E(x)}{
            \Gamma \yields \lambda x.e(x) : \prod_{x:B} E(x)
        }\]
    \item $\Pi$-elimination
        \[
            \frac{\Gamma \yields f:\prod_{x:B}E(x)
            \qquad
            \Gamma \yields b:B
            }{
                \Gamma \yields fb:E(b)
            }
        \]
    \item $\Pi$-computation
        \[
            \frac{
                \Gamma, x:B\yields e(x):E(x) 
                \qquad
                \Gamma \yields b:B
            }
            {
                \Gamma \yields 
                    (\lambda x. e(x))b = e(b) : E(b)
            }
        \]
    \item $\Pi$-uniqueness
        \[ 
            \frac{
                \Gamma \yields f:\prod_{x:B}E(x)
            }{
                \Gamma \yields \lambda x.fx = f : \prod_{x:B}E(x)
            }
        \]
\end{enumerate}

\subsection{$\wedge$-types}

We can form $\wedge$-types out of $\Pi$-types in the exact same 
way as we form $\vee$-types from $\Sigma$-types.

We have 
\[\yields S_0:U\qquad \yields S_1:U, \]
so 
\[b:\BB \yields S(b) :U.\]
Then 
\[\yields \prod_{b:\BB} S(b):U. \]

Then if $\yields s_0:S_0$, $\yields s_1:S_1$, 
$b:\BB \yields s(b):S(b)$, and we have 
\[\yields \lambda b.s(b) : \prod_{b:\BB} S(b).\]

Lastly, we get the $\pi_1$, $\pi_2$ maps as 
\[\frac{\yields p:\prod_{b:\BB} S(b)}{\yields p0:S(0)\qquad \yields p1:S(1)}\]

We'll denote $S_0\wedge S_1$ by $S_0\times S_1$ from now on,
as we're a bit more interested in the set interpretation than
the logic interpretation.

\subsection{$\implies$-types}

Starting with types $\yields S:U$ and $\yields T:U$,
we have $x:S\yields T:U$, and we can form
\[\yields \prod_{x:S} T:U. \]
Then we have
\[\frac{x:S\yields f(x):T}{\yields \lambda x.f(x): \prod_{x:S}T}.\]

From now on, we'll write this as $\to$, since, once again, 
the set interepretation is somehow closer to what we're doing now.

\subsection{Logic interpretation}

To prove $\ds \prod_{x:S} T(x)$, we have to prove $T(x)$ for all
$x:S$. Looks like $\forall_{x:S} T(x)$. ``For all $x\in S$,
$T(x)$ holds.''

As with the $\Sigma$-type/$\exists$ correspondence, this 
product $\forall$ type is somehow stronger than the logic
$\forall$.

Life hack. Read $\Pi$ as $\forall$, and $\Sigma$ as $\exists$.
This will make formulas in type theory make much more sense to 
you.

\section{Returning to $\IdT$-types.}

They form an equivalence relation. A relation on $T$ is 
a dependent type
$s:T,t:T\yields R(s,t):U$.
We can think of a relation as a function $R:T\times T\to U$.

In type theory, and functional programming,
it is often better to instead think about $R:T\to (T\to U)$.
These two types are equivalent in a way that we can't really 
talk about yet.

We can define the type of relations on $T$.
\[\Rel(T) := T\to T\to U. \]
Note that we are not writing parentheses. Functions types 
are assumed to associate to the left.
Then 
\[T:U\yields \Rel(T)=T\to T\to U : U.\]

Given $a:T$ and $b:T$, we have 
\[\IdT_T(a,b) :U. \]
This gives us a dependent type
\[ x:T,y:T\yields \IdT_T(x,y):U. \]
Lambda abstracting gives us 
\[x:T\yields \lambda y. \IdT_T(x,y) : \prod_{y:T}U,\]
and again, we have 
\[\yields \lambda x.\lambda y. \IdT_T(x,y) : \prod_{x:T}\prod_{y:T}U.\]
As mentioned before, we'll write this last type as 
$T\to T\to U=\Rel(T)$. We'll call this term 
$\IdT_T:=\lambda x.\lambda y.\IdT_T(x,y)$.

How do we show that something is reflexive?
We want to define a type $\pred{isRefl}$ so that the type 
being inhabited corresponds to a proof of reflexivity:
% TODO fix?
\[ T:U, R:\Rel(T) \yields \pred{isRefl}(R):U \]
\[s,t:T\yields R(s,t):U\]
\[t:T\yields \refl_t:R(t,t) \yields \lambda t.\refl_t : 
\prod_{t:T}\Rel(t,t).\]
This last type should be our predicate $\pred{isRefl}$:
\[\pred{isRefl}(R):=\prod_{t:T}R(t,t).\]

Then to prove that $\IdT$ is reflexive, we have 
% TODO expand proof here
\[T:U,t:T\yields \lambda t.\refl_t:\prod_{t:T}\IdT(t,t), \]
and this type is precisely 
$\pred{isRefl}(\IdT)$.

Now for symmetry, once again, we want to define a predicate type
$\pred{isSym}(R)$.
Translating from logic, it should be 
\[\pred{isSym}(R):=\prod_{s,t:T} R(s,t)\to R(t,s). \]

Now let's prove that $\IdT_T$ is symmetric.
we start with $s,t:T,p:\IdT_T(s,t) \yields \IdT_T(t,s):U$.
Now 
$t:T\yields \refl_t:\IdT_T(t,t)$, so by the elimination rule
for the identity type,
\[ s,t:T,p:\IdT_T(s,t) \yields j_{\refl_t}(s,t,p): \IdT_T(t,s):U.\]
Then by lambda abstracting, we get
\[\yields \lambda s,t,p. j_{\refl_t}(s,t,p) : 
\prod_{s,t:T}\prod_{p:\IdT_T(s,t)}\IdT_T(t,s)
=\pred{isSym}(R).\]

Finally, we just need to prove transitivity.
This time the type should be 
\[\prod_{r,s,t:T} R(r,s)\to R(s,t)\to R(r,t).\]

Now we prove that the identity type is transitive.
We start with $r,s,t:T$, $p:\IdT_T(r,s)$, $q:\IdT_T(s,t)$,
and we want to get $\IdT_T(r,t)$. 

It suffices to prove
\[
    t:T,r,s:T,p:\IdT_T(r,s)
    \yields 
    ?:\IdT_T(s,t)\to \IdT_T(r,t):U.
\]
\[ t:T\yields  \lambda x.x : \IdT_T(r,t) \to \IdT_T(r,t), \]
so if we have % TODO
\[ t:T,r,s:T,p:\IdT_T(r,s)\yields  j_{\lambda x.x}(t,r,s,p) :
\IdT_T(s,t) \to \IdT_T(r,t) \]
then we can lambda abstract, getting 
\[ \lambda r,s,t,p. j_{\lambda x.x}(t,r,s,p) : 
\prod_{r,s,t} \IdT_T(r,s)\to \IdT_T(s,t) \to \IdT_T(r,t)
=\pred{isTrans}(\IdT_T).
\]

Lastly, we can define an equivalence relation in the type theory,
we can define 
\[T:U,R:\Rel(T) \yields \pred{isEquivRel}(R) 
:= \pred{isRefl}(R)\times \pred{isSym}(R)\times \pred{isTrans}(R):U.\]
Then we have 
\[ T:U\yields ((r,s),t) : \pred{isEquivRel}(\IdT_T).
\]



\end{document}