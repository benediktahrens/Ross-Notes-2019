\documentclass{article}


\title{Type Theory with Paige North 7/22}
\author{Jason Schuchardt}

\usepackage{jragonfyre}
\usepackage{turnstile}
\usepackage{bussproofs}
\usepackage{booktabs}
\newcommand\blank{ \underline{\phantom{$\quad\quad$}} }
\newcommand\TYPE{\text{ \scshape{type}}}
\newcommand\ctxt{\text{ ctxt}}
\newcommand\add{\operatorname{add}}
\newcommand\mult{\operatorname{mult}}
\newcommand\List{\textsc{List}}
\newcommand\nil{\textrm{nil}}
\newcommand\con{\operatorname{con}}
\newcommand\pred[1]{\operatorname{#1}}
\newcommand\isEquiv{\pred{isEquiv}}
%\newcommand\IdT{\mathrm{Id}}
\newcommand\refl{\mathrm{refl}}
\newcommand\BB{\mathbb{B}}
\newcommand\Rel{\operatorname{Rel}}
\newcommand\FunExt{\mathrm{FunExt}}
\newcommand\isContr{\pred{isContr}}
\newcommand\isProp{\pred{isProp}}
\newcommand\isSet{\pred{isSet}}
\newcommand\GroupT{\mathbb{G}\mathrm{roup}}
\newcommand\isnType{\pred{isnType}}

\begin{document}

\maketitle


Last time we talked about contractibility (level 0), and 
we defined 
\[ \isContr(T):= \sum_{t:T} \prod_{s:T} s=t. \]

\section{Propositions ($h$ level 1)}

This time we will talk about propositions.
\begin{definition}
    \[T:U\yields \isProp(T):U, \]
    where 
    \[ \isProp(T) := \prod_{s,t:T} \isContr(s=t).\]
\end{definition}

This roughly means that $T$ is a proposition if and only if 
$T\simeq \bot $ or $T\simeq \top$. (This isn't strictly true. We 
would need law of excluded middle to prove it.)

\begin{proposition}
    If $T\simeq \bot$, then $T$ is a proposition.
    If $t:T $ and $T$ is a proposition, then $T\simeq \top$.
\end{proposition}

\begin{proof}
    We begin by showing $\bot$ is a proposition.
    \[\isProp(\bot) = \prod_{x,y:\bot} \isContr(x=y).\]
    Then by $\bot$-elimination, there is always a dependent function 
    out of bottom into any type. Thus $\isProp(\bot)$ is true.

    Suppose we have $t:T,p:\isProp(T)$.
    We want to construct a pair of a term $t:T$,
    and an element of $\prod_{s:T} s=t$. We already have a 
    $t:T$, and we have 
    \[ p:\isProp(T) :=\prod_{s,t:T} \isContr(s=t). \]
    Then \[ pt : \prod_{s:T} \isContr(s=t).\] Then 
    \[ \lambda s. \pi_1(pts) : \prod_{s:T}s=t \]
    is our desired term.
\end{proof}

Why do we consider propositions?

Consider a dependent type, $b:B\yields E(b)$. For example,
$n:\NN\yields \pred{isEven}(n):U$. Then we might want to define 
the `subtype' of $B$ consisting of all $b:B$ such 
that $E(b)$ is inhabited. 
In set theory, we can just form 
\[\set{b\in B \mid E(b) }.\]
In type theory on the other hand, the closest thing we can 
produce is the $\Sigma$-type:
\[\sum_{b:B} E(b). \]
Then we might define the type of even natural numbers to be 
\[E:= \sum_{n:\NN} \pred{isEven}(n).\]

Now we might run into a problem if our predicate is very
complicated, for example, we might have many proofs of 
$E(b)$, then 
\[\sum_{b:B} E(b)\]
might be very complicated, and not behave like a subtype.

However if each of $E(b)$ is a proposition, then it will 
behave like a subtype should behave.
For example, 
\[n:\NN \yields \pred{isEven}(n) := \sum_{m:\NN} 2m=n .\]
is a proposition.

\begin{proposition}
Consider $b:B\yields E(b)$, together with 
\[b:B\yields p(b) : \isProp(E(b)).\]
If we have 
\[s,t:\sum_{b:B} E(b) \]
such that $\pi_1s=\pi_1t$, then $s=t$.
\end{proposition}

\begin{proof}
    Consider $q:\pi_1s=\pi_1t$. We know 
    \[ s=t
    \simeq
    \sum_{a:\pi_1s=\pi_1t} a_*\pi_2s = \pi_2t
    \]
    We already have $q:\pi_1s=\pi_1t$. We now just need 
    to find something of type $q_*\pi_2s=\pi_2t$.
    This is an identity type between terms of $E(\pi_1t)$.
    However, we know that $E(\pi_1t)$ is a proposition, so 
    $q_*\pi_2s=\pi_2t$ is contractible. It is therefore inhabited
    by some term $c$. Then $(q,c)$ is a term in our sum type.
\end{proof}

Side note, once again, recall that all English statements are 
just translations of type theoretical statements.
The proposition above is the following statement
\[ B:U,
E:B\to U,
p:\prod_{b:B} \isProp(E(b)),
s,t:\sum_{b:B} E(b),
q:\pi_1s=\pi_1t
\yields ?:s=t :U.
\]

\begin{proposition}
    $\isEquiv(f)$, $\isContr(T)$, $\isProp(T)$ are always 
    propositions.
    $\pred{isQEquiv}$ is not a proposition generally. (This is 
    why we say that equivalences are better behaved than 
    quasiequivalences.)
\end{proposition}

\begin{example}
    \[\sum_{f:A\to B} \isEquiv(f) \quad \text{``}\subset\text{''}\quad A\to B\]
\end{example}

\section{Sets ($h$ level 2)}

\begin{definition}
    \[T:U \yields \isSet(T):U,\]
    where 
    \[\isSet(T) := \prod_{s,t:T} \isProp(s=t) \]
\end{definition}

Picture: We think of $T$ as a space, with terms $r,s,t$ being 
points of the space. Then we might have $p,q:s\to t$ paths.
The space of paths is either empty or contractible. So if 
$p$ and $q$ are both paths, then we have a term of the identity 
type $p=q$.

\begin{example}[Groups]
    \[\GroupT := 
    \sum_{G:U}
    \sum_{p:\isSet(G)}
    \sum_{m:G\times G\to G}
    \sum_{e:G}
    \sum_{i:G\to G}
    \]
    \[
    \of*{
    \prod_{a,b,c:G}
    m(a,m(b,c)) = m(m(a,b),c)
    }
    \]
    \[
    \times 
    \of*{
    \prod_{g:G}
    m(g,e) = g \times m(e,g) = g
    }
    \]
    \[
    \times 
    \of*{
        \prod_{g:G}
        m(g,i g) = e
        \times 
        m(i g,g) = e
    }
    \]

    Now we need $G$ to be a set.
    All identity types are propositions. This tells us that 
    the equalities we impose are forming a nice subtype.

    If $G$ were not a set, then we might need equalities between
    our equalities and equalities between all the equality 
    equalities and so on. We'd need infinitely many 
    equalities, which would be bad.
\end{example}

\begin{proposition}
    $\bot$, $\top$, $\BB$, $\NN$ are sets.
\end{proposition}

\section{$h$-levels}

\begin{definition}
$T:U,n:\NN\yields \isnType(n,T):U$, where
\[ \isnType(n,T) :=
\begin{dcases}
    \isnType(0,T) := \isContr(T) &\\
    \isnType(sn,T) := \prod_{s,t:T} \isnType(n,s=t) &
\end{dcases}\]
\end{definition}

\subsection{Topology}

$\pi_0(X)$ is the set of path-connected components.
If $X=\TT^2\sqcup\TT^2$,
then $\pi_0(\TT^2\sqcup \TT^2) = 2$,
and $\pi_1(\TT^2) = \ZZ\times \ZZ$.
We also have $\pi_2(X)$, $\pi_3(X)$, and so on.

Being contractible is $\pi_n(X)=*$.
Being proposition is $\pi_0(X) \in \set{0,1}$ and 
$\pi_n(X)=*$ for $n>0$.
Being a set means $\pi_0$ can be anything, and 
$\pi_n(X)=*$ for $n>0$.
Being a 3-type means $\pi_0(X),\pi_1(X)$ can be anything
$\pi_n(X)=*$ for $n>1$.

This gives us a stratification of types and a way to measure 
their complexity.

\begin{proposition}
    If $T$ is an $n$-type, then $T$ is an $n+1$-type.
\end{proposition}

\begin{proof}
    Induct on $n$.
    $n=0$. Suppose $T$ is a $0$-type, so we have 
    \[ c:\isContr(T) := \sum_{t:T}\prod_{s:T}(t=s).\]
    We need to show 
    \[ \isProp(T):=\prod_{s,s':T} \isContr(s=s').\]
    Fix $s,s'$. Since we have $c=(t,p)$, we get 
    $p(s):t=s$ and $p(s'):t=s'$. Then we have 
    $p(s)\inv \cdot p(s') : s=s'$.
    Now we have 
    \[\prod_{s,s':T} \prod_{q:s=s'} p(s)\inv \cdot p(s') = q.\]
    Then we induct on $q$. Thus we just need to show 
    \[ \prod_{s:T} p(s)\inv \cdot p(s) = \refl_s. \]
    Let's quickly prove $p\inv \cdot p = \refl_s$ 
    for all paths $p$.
    Induct on $p$, then we want to show
    \[ \refl_s\inv \cdot \refl_s = \refl_s,\]
    but this is true by definition of inverses and path 
    composition for reflexivities.

    So we have a term in 
    \[\prod_{s,s':T} \isContr(s=s') =: \isProp(T).\]

    Now the inductive step. Want to show that 
    \[ \isnType(n,T) \to \isnType(sn,T).\]
    By definition, \[\isnType(sn,T) 
    := \prod_{x,y:T} \isnType(n,x=y).\]
    Then by the induction, we have 
    $\ell_{n,x=y} : \isnType(n,x=y) \to \isnType(sn,x=y)$.
    Then composing this with 
    $p:\isnType(n,T)$, we obtain a term of 
    $\isnType(sn,T)$.
\end{proof}

\begin{corollary}
    $\top$ is a proposition, set. $\bot$ is a set.
\end{corollary}

\begin{proposition}
    $\BB$ is a set.
\end{proposition}

\begin{proposition}
    If we have $\BB$ and the univalence axiom, then 
    $U$ is not a set. Indeed, it is not an $n$-type for 
    any $n$.
\end{proposition}

\end{document}